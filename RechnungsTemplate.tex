%%%%%%%%%%%%%%%%%%%%%%%%%%%%%%%%%%%%%%%%%%%%%%%%%%%%%%
%%Dieses Dokument benötigt folgende Dateien im selben Ordern:
%%Produkte.csv und Rechnung.csv
%%Trennung
%%Absender.tex, die alle Informationen über den Absender enthält.
%%%%%%%%%%%%%%%%%%%%%%%%%%%%%%%%%%%%%%%%%%%%%%%%%%%%%%
\documentclass[]{scrlttr2}
\KOMAoptions{
firsthead = on,
backaddress=on,
foldmarks=off,
subject = beforeopening
}
\errorcontextlines 10000
\usepackage[ngerman]{babel}
\usepackage{lmodern}


\usepackage{booktabs}
\usepackage{array}
\usepackage{color,colortbl}
\usepackage{graphics}
\usepackage{multirow}

\usepackage{array}
\usepackage{tabularx}
\usepackage{pgfplotstable}
\pgfplotsset{compat = newest}


\pgfplotstableset{
precision=2,
column type = , %wird von tabularx gemacht
fixed,
 zerofill,
}


%%Sender
\setkomavar{fromname}{} %%Firmenname
\newkomavar{fromname2}
\setkomavar{fromname2}{} %%Firmenzusatz
\setkomavar{fromaddress}{Herner Str.299\\Gebäude B29\\44809 Bochum}
\setkomavar{fromphone}{+49\, (0)234\,}
\setkomavar{fromemail}{} %E-Mail
\setkomavar{fromurl}{} %Internetseite
\setkomavar{signature}{g}
\setkomavar{place}{Bochum}
\setkomavar{date}{\today}
\setkomavar{fromlogo}{\includegraphics[width =0.5\textwidth]{}} %Logo
\setkomavar{subject}{\textbf{\large{Rechnungsnr.\, }}}

%%Fußzeile
\setlength{\tabcolsep}{0.2cm}

\setkomavar{firstfoot}[]{% %%%Der muss angepasst werden!!! je nach firma
\scriptsize{
\begin{tabularx}{\textwidth}{llll}
\multicolumn{4}{l}{\normalsize{Gelieferte Ware bleibt bis zur vollständigen Bezahlung unser Eigentum.}}\\
\multicolumn{4}{l}{ }\\
\usekomavar{fromname} &Bankverbindung&AGB&Geschäftsführer\\
\usekomavar{fromname2} & #1 &Gerichtsstand ist Bochum&#2\\ %#1 Bank, #2 Geschäftsführer Name
Steuer--Nr.&SWIFT/BIC--Code: #3&Erfüllungsort Bochum& \\ %#3 Bix
UST--ID: #4 &IBAN: #5& & \\ %#4 Umsatzsteuer id, #5 Iban der form Dexx\, xxx\,
\end{tabularx}
}
}


%Trennzeichen für die Rücksendeadresee
\setkomavar{backaddressseparator}{ - }
\setkomavar{backaddress}{\usekomavar{fromaddress}}

\newcommand{\absender}[0]{\small{
\begin{tabular}{l} %so ist alles linksbündig, wird aber rechts dargestellt
\textbf{\usekomavar{fromname}}\\
\textbf{\usekomavar{fromname2}}\\
\usekomavar{fromaddress}\\
\rule{0pt}{1.2\baselineskip} \\
\usekomavar{fromphone}\\
\usekomavar{fromemail}\\
\usekomavar{fromurl}\\
\end{tabular}}
}

%%Kopzeile
\setkomavar{firsthead}[]{%
\usekomavar{fromlogo}
\baselineskip
\begin{flushright}
\absender
\end{flushright}
}



%%Empfänger
\ifdefined\firma
\newcommand{\empfaenger}{\small{
\name \\
\strasse \\
\ort \\
\smallskip
\textbf{Lieferadresse:}\\
\textbf{\firma}\\
\firmastrasse \\
\firmaort \\
\medskip
GERMANY
}
}
\else
 \newcommand{\empfaenger}{\small{
name\\
strasse\\
ort\\
GERMANY
}
}
\fi


\begin{document}{
\relax

%\LoadLetterOption{DIN}
\makeatletter
\@addtoplength{firstfootvpos}{-2.5cm}
\@addtoplength{backaddrheight}{0cm}
\@addtoplength{toaddrvpos}{0cm}
\@setplength{toaddrindent}{0.2cm}
\makeatother
\begin{letter}{\empfaenger}
\opening{}

%Tabelle, die Rechnungsposten enthalten soll
\pgfplotstabletypeset[
begin table={\begin{tabularx}{\textwidth}{lXcrr}},
end table={\end{tabularx}},
col sep = space,
%row sep = \\,
%read comma as period = true,
columns={Position, Posten, Anzahl, Einzelpreis, Gesamtpreis},
columns/Posten/.style={string type},
columns/Position/.style={precision=0},
columns/Anzahl/.style={precision=0},
assign column name/.style={
	/pgfplots/table/column name={\textbf{#1}}}
	,
every head row/.style= {
	after row = \toprule},
every last row/.style={
	 after row =\bottomrule},
every odd row/.style ={
	before row={\rowcolor{lightgray}}}
]{Produkte.csv}

%Tabelle mit Endbeträgen
\raggedleft
\vspace*{1.5cm} %%Abstand zwischen Rechnungsposten und Endsumme

\pgfplotstabletypeset[
col sep =&,
columns={bez, betrag},
begin table={\begin{tabularx}{0.35\textwidth}{Xr}},
end table={\end{tabularx}},
columns/bez/.style={string type},
columns/betrag/.style={precision = 2},
assign column name/.style={
	/pgfplots/table/column name={\textbf{#1}}}
	,
every head row/.style= {output empty row},
every first row/.style ={
	after row = \toprule},
last head row/.style = {
	int type, precision = 2},
every last row/.style={
	%{\textbf{#1}}
	 after row =\bottomrule}
]{Rechnung.csv}


%Zahlungsmethode

\end{letter}}
\end{document}